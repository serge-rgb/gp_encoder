\chapter{Introducción}\label{ch:introduction}


% background
% ==== una pagina
La Compresión de Datos es el acto de transformar la respresentación de la
información con el propósito de reducir su tamaño.

Se le llama \emph{compresión sin pérdida} a la compresión que se hace de manera
que la información representada se mantenga intacta. Por otro lado, la
\emph{compresión con pérdida} es una transformación que reduce el tamaño de la
representación y que permite un grado de pérdida de información.

Un ejemplo de compresión es el uso de la multiplicación para escribir sumas
repetidas de una manera más corta, o de la exponenciación, que comprime de la
misma manera la multiplicación repetida.

$ a + a + a + a + a $ se comprime con $ 5 * a $. Ambas ecuaciones contienen la
misma información, el acto de comprimir reduce el tamaño de la
\emph{representación}, dejando intacto el significado.

La Compresión de Datos es una de las ramas más viejas de las Ciencias de la
Computación. Está sentada sobre fundamentos matemáticos solidificados durante
el mismo periodo en el cual las Ciencias de la Computación se establecían como
un campo legítimo de estudio \cite{cs_the_discipline}.

Es un matrimonio entre teoría y práctica que combina el rigor matemático de las
teorías en las que se basa y el ingenio y la eficacia del implementador.

\section{Motivación}

A principios de la década de 1990, cuando la popularidad de Internet estaba
iniciando su crecimiento exponencial, un comité de matemáticos e ingenieros,
llamado "Joint Photographic Experts Group", o JPEG estandarizó un formato de
compresión de imágenes \cite{jpeg-spec}. El formato JPEG, soportando varios
métodos de compresión, fue adaptado rápidamente y hoy en día es soportado por
prácticamente todo programa que tenga que leer imágenes. Imágenes JPEG están
incluidas en millones de páginas web y virtualmente todo navegador lo soporta.

Por su naturaleza, los formatos de datos son esclavos a la inercia. Mientras
nuestro nuevo conocimiento matemático nos da herramientas que podemos utilizar
para crear técnicas más eficientes y compresión de mayor calidad, los datos
viejos siguen estando en sus viejos formatos y migrar puede ser difícil o
imposible. En el caso de compresión con pérdida, migrar datos a nuevos formatos
hace que se pierda parte de la información. Ésta pérdida, en muchos casos, es
un precio demasiado alto que pagar para obtener las ventajas de nuevos y
mejores formatos con compresión de datos.

Como resultado, se puede notar que la adopción de nuevas técnicas de compresión
tiende a ir de la mano con la migración a nuevos medios tecnológicos, e.g. VHS
a DVD. Es poco probable que en el corto o mediano plazo veamos una migración
tecnológica que nos haga cambiar fundamentalmente la manera que guardamos y
compartimos fotografías. JPEG es la técnica de compresión con pérdida es más
utilizada en el mundo y no hay razón para pensar que eso va a cambiar en el
futuro cercano.

Dado que las computadoras son mucho más rápidas de lo que eran en los 90s, y
dada la dificultad de adoptar nuevos formatos de compresión, la motivación de
este trabajo viene de la pregunta:  ¿Qué se puede hacer con el poder de sobra
que tenemos a nuestra disposición?

El resultado este trabajo es una manera de obtener una compresión JPEG de mejor
calidad a la de codificadores convencionales, usando un algoritmo genético.

\section{Objetivo}

La compresión JPEG, que será descrita en el capítulo \ref{ch:jpeg_desc}, tiene
un componente importante para el cual no existe solución óptima. El componente
al que me refiero es un par de matrices cuadradas de tamaño $8\times8$. La
elección de qué matrices elegir para comprimir una imagen determina en gran
parte la calidad y la tasa de compresión.

La especificación JPEG \cite{jpeg-spec} incluye matrices de ejemplo, pero la
elección de qué par de matrices usar está en las manos de cada codificador
JPEG. En la mayoría de los casos, cada implementador deja fijas en su
codificador una lista de pares de matrices que corresponden a diferentes
niveles de calidad. Las matrices se generan \"a ojo\". Sabiendo cómo funciona
el algoritmo, y probando la calidad de la compresión para cada matriz sobre un
conjunto fijo de imágenes que se consideran representativas.

El enfoque de este trabajo es producir un codificador JPEG que encuentre un par
de matrices ideal para la imagen particular que se está decidiendo codificar.
Se implementó un codificador JPEG diseñado desde cero para adaptarse a un
algoritmo genético. La implementación hace uso del paralelismo en CPUs y GPUs.

En los siguientes capítulos, se da una descripción del funcionamiento de la
compresión JPEG y una introducción a los algoritmos genéticos. Después, se da
una descripción de los detalles de la implementación, seguido de un desglose de
los resultados y una sección donde se habla de las conclusiones obtenidas.


