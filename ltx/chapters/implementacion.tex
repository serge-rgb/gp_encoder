\chapter{Imlementación JPEG}\label{ch:implementacion}

Se implementó un codificador JPEG, lanzado al dominio público en github bajo el
nombre de TinyJPEG \cite{tiny_jpeg}. La ubicación permanente de esta biblioteca
está en \begin{alltt}https://github.com/serge-rgb/TinyJPEG \end{alltt}

Encima de TinyJPEG, se desarrolló un proyecto que extiende TinyJPEG para
implementar el algoritmo evolutivo \cite{gp_encoder}, localizado en
\begin{alltt}https://github.com/serge-rgb/gp_encoder\end{alltt}

En este capítulo se describen los detalles de ambos proyectos.

\section {Términos}
Se van a usar los términos \emph{debugging} y \emph{profiling} por su ubiquidad
en la literatura. \emph{debugging} se traduce como depurar y se define como el
proceso de encontrar y arreglar defectos de código. \emph{profiling} es el
proceso de encontrar los puntos en los que un programa puede ser modificado
para mejorar el desempeño.

\emph{General Purpose GPU}, o \emph{GPGPU} es el
término usado para la práctica de usar programar GPUs directamente, a
diferencia de el uso original, en el cual el GPU era un acelerador cuya
interfaz era una biblioteca de gráficas como OpenGL o DirectX.

% ============================================================
\section{Lenguajes}
% ============================================================

La elección de lenguaje para TinyJPEG fue C, en particular C99 \cite{c99}.
TinyJPEG fue lanzado con el objetivo de ser una biblioteca reutilizable por
otras personas, y ha encontrado cierto grado de éxito. Un planetario en París
usa TinyJPEG para capturar vídeo de sus simulaciones.

C es un lenguaje ideal para escribir cosas como codificadores. El lenguaje
permite mantener el nivel bajo de abstracción que se necesita y las
herramientas de \emph{debugging} son mejores para C y C++ que para casi
cualquier otro lenguaje.

Escogí OpenCL para la implementación de GPGPU. OpenCL es un estándar abierto
para GPGPU y está soportado por Intel, Nvidia y AMD. La máquina en la que se
implementó este trabajo tiene una tarjeta Nvidia. Nvidia tiene su propio
lenguaje para GPGPU, llamado CUDA, y tiene mejor soporte para debugging y
profiling que OpenCL para sus tarjetas de video. Sin embargo, las herramientas
siguen siendo primitivas comparadas con lo que se tiene en el CPU. En cualquier
caso, uno termina haciendo hipótesis, experimentos y medidas para optimizar la
solución, aún teniendo herramientas sofisticadas. Se escogió OpenCL porque los
méritos relativos de facilidad de desarrollo no le ganan al soporte
multi-plataforma y al valor de apoyar estándares abiertos.

% ============================================================
\section{Arquitectura}
% ============================================================

TinyJPEG es minimalista. Es un sólo archivo de poco más de 1000 líneas. Está escrito en el estilo popularizado por Sean Barrett de escribir ún solo archivo \verb+.h+ con la siguiente estructura:

\label{alg:stb}
\begin{code}[language=C][h]
    // Principio del archivo
    #pragma once

    // Definición de la interfaz.
    tje_encode_jpeg(...);

    #ifdef TJE_IMPLEMENTATION

    // La implementación completa va aquí.
\end{code}

De esta manera, uno puede incluir \verb+#include <tiny_jpeg.h>+ como cualquier \emph{header} de C, pero en uno de los archivos del proyecto, se hace esto:


\label{alg:stb_impl}
\begin{code}[language=C][h]
    #define TJE_IMPLEMENTATION
    #include <tiny_jpeg.h>
\end{code}



% ============================================================
\section{Algoritmo DCT}
% ============================================================

A partir de la ecuacion \ref{eq:dct} se puede derivar directamente un algoritmo
simple para la \emph{DCT}:

\label{alg:dct}
\begin{code}[language=C][h]
    float DCT[64];
    for (int v = 0; v < 8; ++v) {
        for (int u = 0; u < 8; ++u) {
            DCT[v*8 + u] = F(u, v);
            // F es la traducción directa de definición DCT
    }
\end{code}


\verb+tiny_jpeg+ contiene dos implementaciones de \emph{DCT}, la que se deriva
directamente de la ecuación \ref{eq:dct} y que se describe en \ref{alg:dct}
muestra arriba, y una más rápida, desarrollada por \cite{ahmed_dct}.

Los algoritmos rápidos para calcular la \emph{DCT} están basados en la
observación de que la ecuación \ref{eq:dct} es lineal, y por lo tanto el
cálculo \emph{DCT} se puede ver como una multiplicación de dos matrices de
$8\times8$. La matriz de la derecha es el bloque con los valores de la imagen y
la matriz de la izquierda es la representación como matriz de la ecuación
\ref{eq:dct}.

% ============================================================
\section{Manejo de memoria}
% ============================================================

% ============================================================
\section{Optimización}
% ============================================================

% ============================================================
\section{GPGPU}
% ============================================================
