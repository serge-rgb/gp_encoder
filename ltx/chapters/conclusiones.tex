
%************************************************
\chapter{Conclusiones}\label{ch:conclusiones}
%************************************************

El proyecto presentado en esta tesis aprovecha los saltos tecnológicos que se
han dado desde la creación de JPEG en 1992 \cite{jpeg-spec} para lograr mejor
resultados usando un formato ya existente. El algoritmo produce buenos
resultados consistentemente, y ocasionalmente consigue simultáneamente una
mejor calidad de imagen y una mayor compresión.

Se hizo mención en un par de ocasiones a que el programa está diseñado para
producir imágenes \emph{indistinguibles} de las originales, sin mencionar el
hecho de que es imposible medir las características cualitativas del sistema
visual humano. Se utilizó una métrica estándar de diferencia \emph{pixel por
pixel} para determinar el error de imagen, pero esta no es la única opción
disponible. Existe un área de investigación dedicada a medir la calidad de
imágenes tomando en cuenta modelos basados en el sistema visual humano
\cite{subjective-paper}.

% Nuevos formatos (Mr. Bellard) que pretenden reemplazar a JPEG.
Recientemente ha habido esfuerzos para reemplazar a JPEG en la web,
notablemente por Google (\cite{brotli}, \cite{webp}). El mayor problema a
resolver es el de inercia. Google usa su poder sobre el Internet, pero esta no
es la única estrategia. Francis Bellard, creador de Qemu, FFmpeg y otros
proyectos de código libre, lanzó un formato llamado \gls{BPG}, cuyo
decodificador está escrito en Javascript, y por lo tanto tiene una adopción
virtualmente universal.

Sin embargo, JPEG se va a mantener relevante. Usar la los recurso de las
máquinas comunes de hoy en día también es una buena estrategia para obtener una
mejora sobre un formato que ya está en la médula del mundo del software.

% Distribución de programas que usan OpenCL
Se utilizó OpenCL para la paralelización en GPUs por ser un formato abierto y
por que actualmente está adoptado por todos los proveedores de cómputo
vectorial masivo (Intel, NVidia y AMD). Actualmente es difícil distribuir de
manera binaria un codificador que use OpenCL. Se tiene la esperanza de que en
el futuro exista una manera inmediata de utilizar las capacidades de cómputo
vectorial que son cada vez más ubicuas.

En las últimas dos décadas ha habido una tendencia hacia una convergencia entre
GPU y CPU. Aunque hoy en día se tiene que escribir código en lenguajes
propietarios o con herramientas primitivas, es claro hacia donde se está
moviendo la marea. Por el lado del hardware, Intel y Nvidia están dando
soluciones para cómputo de alto desempeño con sus productos Xeon Phi y Tesla,
respectivamente. En el lado del consumidor, los procesadores de escritorio de
Intel dedican una porción cada vez más grande de los transistores disponibles a
una unidad de gráficas integradas, que hoy en día puede ser aprovechada con
OpenCL. Los GPUs, por otro lado están, generación tras generación, volviéndose
más programables. Cada día hay más información disponible para el programador
que quiera usar el poder de los GPUs.

Este proyecto aprovechó el hecho de que JPEG cuenta con un problema para el
cual no existe solución, y fue exitoso en lograr su propósito mediante el uso
de \gls{Cómputo Evolutivo}. Los algoritmos genéticos fueron una gran herramienta
para conseguir una solución heurística al problema de las tablas de
cuantificación.

Los algoritmos genéticos se prestan a ser programados con \gls{GPGPU}. La
necesidad de alto rendimiento y la tendencia a que no importe tener alta
latencia sugieren que los GPUs pueden ser usados. Si la función de aptitud es
una función pura y el problema es demandante, entonces no hay razón para no
hacer una implementación en el GPU.
