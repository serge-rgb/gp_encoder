\chapter{Introducción}\label{ch:introduction}


\section{Motivación}

% background
La compresión de datos ha sido un tema importante desde antes de que la revolución informática. Un ejemplo temprano es la invención de la clave Morse para facilitar la comunicación telegráfica. A principios de la década de 1950, David Huffman inventó lo que conocemos como Codificación Huffman, cuando la Teoría de la Información estaba naciente. De la Teoría de la Información crecería una subrama, la Teoría de Codificación, dando inicio al estudio formal de la compresión de datos.
A principios de la década de 1990, cuando la popularidad de Internet estaba iniciando su crecimiento exponencial, un comité de matemáticos e ingenieros, llamado "Joint Photographic Experts Group", o JPEG logró estandarizar un formato de imágen robusto. El formato JPEG, soportando varios métodos de compresión, fue adaptado rápidamente y hoy en día es soportado por prácticamente todo programa que soporte imágenes, incluyendo navegadores, manejadores de archivos, editores fotográficos,
% los formatos de compresión, por que es interesante para un programador
% factores externos que solidifican formatos. inercia: soporte externo. como esto impide el progreso
% descripcion breve de jpeg. hacer nota de que hay una parte que se presta a heurísticas
% explicar que jpeg se creó en un mundo distinto. hoy en día tiene sentido gastar los recursos que tenemos en estas heuristicas

\section{Objetivos}

\section{Estructura de la Tesis}
