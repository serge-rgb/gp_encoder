%*******************************************************
% Abstract
%*******************************************************
%\renewcommand{\abstractname}{Abstract}
\pdfbookmark[1]{Resumen}{Resumen}
\begingroup
\let\clearpage\relax
\let\cleardoublepage\relax
\let\cleardoublepage\relax

\chapter*{Resumen}

En esta tesis se describe el desarollo de un proyecto que logra generar una
mejora sobre el algoritmo de compresión JPEG estándar. Esto se logra mediante
el uso de una heurística individual para la imagen a comprimir. El método
heurístico es un \gls{algoritmo evolutivo} que usa una aproximación de JPEG
para su función de selección. Esta aproximación es una versión paralela de JPEG
que evita realizar trabajo que no es necesario para los propósitos de la
evolución. Se implementa en el CPU una versión secuencial y una paralela, así
como una implementación en el GPU usando OpenCL. Se le da un cuidado especial
al desempeño de la implementación y se describen las técnicas usadas para hacer
tanto optimización algorítmica como micro-optimización.


\vfill

\endgroup

%%% Local Variables:
%%% mode: latex
%%% ispell-local-dictionary: "espanol"
%%% TeX-engine: xetex
%%% TeX-master: "../Tesis_RGC"
%%% End:
